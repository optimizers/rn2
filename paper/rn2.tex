\RequirePackage{fix-cm}
%
%\documentclass{svjour3}                     % onecolumn (standard format)
%\documentclass[smallcondensed]{svjour3}     % onecolumn (ditto)
\documentclass[smallextended]{svjour3}       % onecolumn (second format)
%\documentclass[twocolumn]{svjour3}          % twocolumn
%
\usepackage{springerpaper}
%
% For the Cahier du GERAD.
%
% \renewcommand{\year}{2022}     % If you don't want the current year.
\newcommand{\cahiernumber}{00}  % Insert your Cahier du GERAD number.
%
% Meta-information for the PDF file generated..
%
\newcommand{\papertitle}{A Proximal Modified Quasi-Newton Method for Nonsmooth Regularized Optimization}
\hypersetup{
  pdftitle={\papertitle},
  pdfauthor={Youssef Diouane and Mohamed Laghdaf and Dominique Orban},
  pdfsubject={Nonsmooth Regularized Optimization},
  pdfkeywords={},
}
%
% \usepackage{mathptmx}      % use Times fonts if available on your TeX system
%
% insert here the call for the packages your document requires
%\usepackage{latexsym}
% etc.
%
% please place your own definitions here and don't use \def but
% \newcommand{}{}
\spdefaulttheorem{problemassumption}{Problem Assumption}{\bf}{\it}
\spdefaulttheorem{modelassumption}{Model Assumption}{\bf}{\it}
\spdefaulttheorem{stepassumption}{Step Assumption}{\bf}{\it}
\crefname{problemassumption}{Problem Assumption}{Problem Assumptions}
\Crefname{problemassumption}{Problem Assumption}{Problem Assumptions}
\crefname{modelassumption}{Model Assumption}{Model Assumptions}
\Crefname{modelassumption}{Model Assumption}{Model Assumptions}
\crefname{stepassumption}{Step Assumption}{Step Assumptions}
\Crefname{stepassumption}{Step Assumption}{Step Assumptions}
% No need to use "sub" in front of a subsection reference.
\Crefname{subsection}{Section}{Sections}
\crefname{subsection}{section}{sections}
%
% Insert the name of "your journal" with
\journalname{JOTA}
%
\begin{document}

\title{%
  \papertitle
  \thanks{Research supported by an NSERC Discovery grant.}%
}
% \subtitle{%
%   Do you have a subtitle?\\ If so, write it here
% }

\titlerunning{A Proximal Modified Quasi-Newton Method}        % if too long for running head

\author{Youssef Diouane \and
        Mohamed Laghdaf \and
        Dominique Orban
}

\authorrunning{Diouane, Laghdaf, Orban} % if too long for running head

\institute{%
  Y. Diouane \at
  Department of Mathematics and Industrial Engineering, Polytechnique Montr\'eal.
  \email{youssef.diouane@polymtl.ca}
  \and
  M. Laghdaf \at
  École Nationale Supérieure des Techniques Avancées, Paris
  \email{mohamed.habiboullah@ens-paris-saclay.fr}
  \and
  D. Orban \at
  GERAD and Department of Mathematics and Industrial Engineering, Polytechnique Montr\'eal.
  \email{dominique.orban@gerad.ca}
}

\date{Received: date / Accepted: date}
% The correct dates will be entered by the editor

\linenumbers
\pagestyle{myheadings}

\maketitle
\thispagestyle{mytitlepage}

\begin{abstract}
  We develop method RN2, a modified quasi-Newton method for minimizing the sum of a smooth \(f\) with Lipschitz gradient and lower semi-continuous prox-bounded \(h\).
  Both \(f\) and \(h\) may be nonconvex and may be bound constrained.
  At each iteration, our method computes a step by minimizing the sum of a convex quadratic quasi-Newton model of \(f\), a model of \(h\), and an adaptive quadratic regularization term. 
  A step may be computed by way of methods R2 \citep{aravkin-baraldi-orban-2022} or TRDH \citep{leconte-orban-2023}.
  In variant RN2-DH, the model of \(f\) is diagonal, which allows us to compute a step without resort to a subproblem solver for a few separable \(h\) that are relevant in applications.
  RN2-DH can also be used as subproblem solver inside RN2.
  We establish global convergence of a first-order stationarity measure to zero and a worst-case evaluation complexity bound of \(\mathcal{O}(\epsilon^{-2})\) to bring said measure below \(\epsilon \in (0, \, 1)\).
  We describe our Julia implementation and report numerical experience on inverse problems, and a minimum-rank matrix completion problem.
  \smarttodo{finish abstract}
  \keywords{%
    Nonsmooth optimization \and
    Nonconvex optimization \and
    Regularized optimization \and
    Composite optimization \and
    Modified quasi-Newton method \and
    Proximal quasi-Newton method \and
    Proximal gradient method
  }
% \PACS{PACS code1 \and PACS code2 \and more}
% \subclass{MSC code1 \and MSC code2 \and more}
\end{abstract}

% Pour le cahier du GERAD.
%\begin{resume}
%
%\end{resume}
%\textbf{Mots cl\'es :}

\section{Introduction}%
\label{sec:introduction}

We consider problems of the form
\begin{equation}%
  \label{eq:nlp}
  \minimize{x \in \R^n} \ f(x) + h(x) \quad \st \ \ell \leq x \leq u,
\end{equation}
where \(\ell \in (\R \cup \{-\infty\})^n\), \(u \in (\R \cup \{+\infty\})^n\), \(f: \R^n \to \R\) is differentiable over the box \([\ell, u]\), \(h: \R^n \to \R \cup \{+\infty\}\) is lower semi-continuous, and inequalities are understood componentwise.

Our motivation for developing a modified Newton variant of the trust-region algorithms of \citet{aravkin-baraldi-orban-2022} and \citet{leconte-orban-2023} is that the proximal operators used in the subproblems should be easier to derive as they do not need to account for an extra trust region indicator.
Of course, in the presence of bound constraints, they have to account for a box indicator.

We also develop a variant of the complexity result of \citet{leconte-orban-2023} for the case where the Hessian approximation of \(f\) is not uniformly bounded.

\subsection*{Contributions and Related Research}

\begin{itemize}
  \item \citet{aravkin-baraldi-orban-2022}: trust-region method;
  \item \citet{kanzow-lechner-2021}: nonsmooth term must be convex;
  \item \citet{leconte-orban-2023}: model with diagonal Hessian, trust-region method;
  \item \citet{aravkin-baraldi-orban-2022b}: nonlinear least-squares;
\end{itemize}

\subsection*{Notation}

\section{Background}%
\label{sec:background}

\begin{itemize}
  \item Fréchet and limiting subdifferentials;
  \item proximal operator;
  \item constraint qualification (because of the bounds);
  \item first-order stationarity;
\end{itemize}

\section{Models}%
\label{sec:models}

For \(\nu > 0\), \(\sigma \geq 0\), \(x \in [\ell, u]\), and \(B = B^T \in \R^{n \times n}\) with \(B \succeq 0\), let
\begin{subequations}
  \begin{align}
    \varphi_1(s; x, \nu^{-1}) & := f(x) + \nabla f(x)^T s + \tfrac{1}{2} \nu^{-1} \|s\|^2 \\
    \varphi(s; x, \sigma)     & := f(x) + \nabla f(x)^T s + \tfrac{1}{2} s^T (B + \sigma I) s \\
    \psi(s; x)                & \phantom{:}\approx h(x + s) \\
    m_1(s; x, \nu^{-1})       & := \varphi_1(s; x, \nu^{-1}) + \psi(s; x) + \chi(s \mid [\ell - x, u - x]) \\
    m(s; x, \sigma)           & := \varphi(s; x, \sigma) + \psi(s; x) + \chi(s \mid [\ell - x, u - x]).
  \end{align}
\end{subequations}

\begin{modelassumption}%
  \label{asm:psi}
  For any \(x \in \R^n\), \(\psi(\cdot; x)\) is proper, lsc and prox-bounded with threshold \(\lambda_x \in \R_+ \cup \{+\infty\}\).
  In addition, \(\psi(0; x) = h(x)\), and \(\partial \psi(0; x) = \partial h(x)\).
\end{modelassumption}

Define also
\begin{subequations}
  \begin{align}
    p_1(x; \nu^{-1})   & := \min_s \ m_1(s; x, \nu^{-1}) \leq m_1(0; x, \nu^{-1}) = f(x) + h(x) \\
    P_1(x; \nu^{-1})   & := \argmin{s} \ m_1(s; x, \nu^{-1}) \\
    \xi_1(s; \nu^{-1}) & := f(x) + h(x) - p_1(x; \nu^{-1}) \geq 0 \label{eq:def-xi1}.
  \end{align}
\end{subequations}

\section{Algorithm}%
\label{sec:algorithm}

\begin{algorithm}[ht]
  \caption[caption]{%
    Nonsmooth regularized modified Newton method.%
    \label{alg:RN2}
  }
  \begin{algorithmic}[1]
    \State Choose constants \(0 < \eta_1 \leq \eta_2 < 1\) and \(0 < \gamma_3 \leq 1 < \gamma_1 \leq \gamma_2\).
    \State Choose \(x_0 \in \R^n\) where \(h\) is finite, \(\sigma_0 > 0\), compute \(f(x_0)\) and \(h(x_0)\).
    \For{\(k = 0, 1, \dots\)}
      \State\label{alg:RN2:Bk}%
      Choose \(B_k = B_k^T \in \R^{n \times n}\) with \(B_k \succeq 0\).
      \State\label{alg:RN2:step-nuk}%
      Choose a steplength \(\nu_k < 1 / (\|B_k\| + \sigma_k)\).
      \State Compute \(s_{k,1} \in \argmin{s} \ m_1(s; x_k, \nu_k)\) and \(\xi_1(x_k, \nu_k)\) as defined in~\eqref{eq:def-xi1}.
      \State\label{alg:RN2:step-computation}%
      Compute an approximate minimizer \(s_k\) of \(m(s; x_k, \sigma_k)\).
      \State\label{alg:RN2:step-rhok}%
      Compute the ratio
      \[
      \rho_k :=
      \frac{
        f(x_k) + h(x_k) - (f(x_k + s_k) + h(x_k + s_k))
      }{
        \varphi(0; x_k) + \psi(0; x_k) - (\varphi(s_k; x_k) + \psi(s_k; x_k))
      }.
      \]
      \State\label{alg:RN2:step-accept}%
      If \(\rho_k \geq \eta_1\), set \(x_{k+1} = x_k + s_k\). Otherwise, set \(x_{k+1} = x_k\).
      \State\label{alg:RN2:step-update}%
      Update the regularization parameter according to
      \[
        \sigma_{k+1} \in
        \begin{cases}
          [\gamma_3 \sigma_k, \, \sigma_k] & \text{ if } \rho_k \geq \eta_2,
          \\ [\sigma_k, \, \gamma_1 \sigma_k] & \text{ if } \eta_1 \leq \rho_k < \eta_2,
          \\ [\gamma_1 \sigma_k, \, \gamma_2 \sigma_k] & \text{ if } \rho_k < \eta_1.
        \end{cases}
      \]
    \EndFor
  \end{algorithmic}
\end{algorithm}

Unlike \citep[Algorithm~\(3.1\)]{aravkin-baraldi-orban-2022}, \Cref{alg:RN2} requires \(B_k \succeq 0\), and thus may fail to model negative curvature that may be naturally present in \(f\).
However, proximal operators should be easier to derive than in the case of \citep[Algorithm~\(3.1\)]{aravkin-baraldi-orban-2022} as they do not involve an extra indicator for the trust region.

\begin{problemassumption}%
  \label{asm:g-lipschitz}
  \(f\) is continuously differentiable on \(\{x \in [\ell, u] \mid (f+h)(x) \leq (f+h)(x_0)\}\) and \(h\) is proper and lower semi-continuous.
\end{problemassumption}

\begin{remark}
  We do not require \(\nabla f\) to be Lipschitz continuous.
\end{remark}

\section{Convergence Properties and Evaluation Complexity}%
\label{sec:convergence}

The convergence properties of \Cref{alg:RN2} mirror those of \citep[Algorithm~\(4.1\)]{aravkin-baraldi-orban-2022b} and the proofs are identical.
We omit them below and refer the interested reader to \citep{aravkin-baraldi-orban-2022b} for complete details.

\begin{modelassumption}%
  \label{asm:model-adequacy}
  There is \(\kappa_{\textup{m}} > 0\) such that for all \(x\) and \(s \in \R^n\), \(|(f + h)(x + s) - (\varphi +
\psi)(s; x)| \leq \kappa_{\textup{m}} \|s\|^2\).
\end{modelassumption}

We make the following additional assumption and say that \(\{\psi(\cdot; x_k)\}\) is \emph{uniformly prox-bounded}.

\begin{modelassumption}%
  \label{asm:unif-prox-bounded}
  There exists \(\lambda > 0\) such that \(\lambda_{x_k} \geq \lambda\) for all \(k \in \N\).
\end{modelassumption}

\Cref{asm:unif-prox-bounded} is satisfied if \(h\) itself is prox-bounded, and we select \(\psi(s; x_k) := h(x_k + s)\) at each iteration.

\begin{theorem}%
  \label{thm:sigmasucc}
  Let \Cref{asm:psi,asm:model-adequacy,asm:unif-prox-bounded} be satisfied, and let
  \begin{equation}
    \label{eq:def-sigmasucc}
    \sigma_{\textup{succ}} :=
    \max(2 \kappa_{\textup{m}} / (1 - \eta_2), \, \lambda^{-1})
     > 0.
  \end{equation}
  If \(x_k\) is not first-order stationary and
  \(
    \sigma_k \geq \sigma_{\textup{succ}}
  \),
  then iteration \(k\) is very successful and
  \(
    \sigma_{k+1} \leq \sigma_k
  \).
\end{theorem}

By \Cref{thm:sigmasucc}, there exists \(\sigma_{\max} > 0\) such that
\begin{equation}%
  \label{eq:sigmamax}
  \sigma_k \leq
  \sigma_{\max} := \min(\sigma_0, \gamma_2 \sigma_{\textup{succ}}) > 0
  \quad \text{for all } k \in \N.
\end{equation}

\begin{theorem}%
  \label{thm:lm-finite}
  Let \Cref{asm:g-lipschitz,asm:psi,asm:model-adequacy} be satisfied.
  If \Cref{alg:RN2} only generates finitely many successful iterations, then \(x_k = x^*\) for all sufficiently large
\(k\) and \(x^*\) is first-order critical.
\end{theorem}

Let \(0 < \theta < 1\).
In Step~\ref{alg:RN2:step-nuk} of \Cref{alg:RN2}, we set
\begin{equation}%
  \label{eq:nuk}
  \nu_k := \theta / (\|B_k\| + \sigma_k).
\end{equation}

\subsection{Bounded Hessian Approximations}%
\label{sec:B-bounded}

\begin{modelassumption}%
  \label{asm:B-bounded}
  There exists \(\kappa_B > 0\) such that \(\|B_k\| \leq \kappa_B\) for all \(k \in \N\).
\end{modelassumption}

\Cref{asm:B-bounded} and~\eqref{eq:sigmamax} yield
\begin{equation}%
  \label{eq:numin}
  \nu_k \geq \theta / (\kappa_B + \sigma_{\max}) := \nu_{\min} > 0
  \quad \text{for all } k \in \N.
\end{equation}
Therefore, \(\nu_k^{-1} \leq \nu_{\min}^{-1}\) for all \(k \geq 0\), and \citep[Theorem~\(1.25\)]{rtrw} implies
\begin{equation}%
  \label{eq:ximin-nu}
  \xi_1(x_k, \nu_k^{-1}) \geq \xi_1(x_k, \nu_{\min}^{-1})
  \quad \text{for all } k \in \N.
\end{equation}

For \(\epsilon \in (0, \, 1)\), we seek \(k(\epsilon) \in \N\) such that
\begin{equation}
  \label{eq:RN2-stop}
  \xi_1{(x_k, \nu_{\min}^{-1})}^{\frac12} > \epsilon \quad \text{for all } k < k(\epsilon)
  \quad \text{and} \quad
  \xi_1{(x_{k(\epsilon)}, \nu_{\min}^{-1})}^{\frac12} \leq \epsilon.
\end{equation}
Define the sets
\begin{subequations}%
  \label{eq:S-U-sets}
  \begin{align}
       \mathcal{S} & := \{ k \in \N \mid \rho_k \geq \eta_1 \},
    \\ \mathcal{S}(\epsilon) & := \{ k \in \mathcal{S} \mid k < k(\epsilon) \},
    \\ \mathcal{U}(\epsilon) & := \{k \in \N \mid k \not \in \mathcal{S} \text{ and } k < k(\epsilon) \}.
  \end{align}
\end{subequations}

At this point, it is necessary to assume that the step computed at step~\ref{alg:RN2:step-computation} of \Cref{alg:RN2} is related to \(\xi_1(x_k, \sigma_k)\).
That is the subject of the next assumption.

\begin{stepassumption}%
  \label{asm:step-computation}
  There exists \(\kappa_{\textup{mdc}} \in (0, \, 1)\) such that \(s_k\) computed at stage~\ref{alg:RN2:step-computation} of \Cref{alg:RN2} satisfies
  \begin{equation}%
    \label{eq:sufficient-decrease}
    \varphi(0; x_k) + \psi(0; x_k) - (\varphi(s_k; x_k) + \psi(s_k; x_k)) \geq
    \kappa_{\textup{mdc}} \xi_1(x_k, \nu_k^{-1}).
  \end{equation}
\end{stepassumption}

\begin{theorem}%
  \label{thm:lm-complexity-bound}
  Let \Cref{asm:g-lipschitz,asm:psi,asm:model-adequacy} be satisfied and \(s_k\) be computed according to \Cref{asm:step-computation}, where \(\nu_k\) is chosen according to~\eqref{eq:nuk}.
  Assume there are infinitely many successful iterations and that \(f(x) + h(x) \geq {(f + h)}_{\textup{low}}\) for all \(x \in \R^n\).
  Then,
  \begin{equation}
    \label{eq:lm-bound-Seps}
    |\mathcal{S}(\epsilon)| \leq
    \frac{(f + h)(x_0) - (f + h)_{\textup{low}}}{\eta_1 \kappa_{\textup{mdc}} \epsilon^2} =
    O(\epsilon^{-2}),
  \end{equation}
  and
  \begin{equation}
    \label{eq:lm-bound-Ueps}
    |\mathcal{U}(\epsilon)| \leq
    \frac{\log(\sigma_{\max} / \sigma_0)}{\log(\gamma_1)} + |\mathcal{S}(\epsilon)| \frac{|\log(\gamma_3)|}{\log(\gamma_1)}
    =
    O(\epsilon^{-2}).
  \end{equation}
\end{theorem}

\subsection{Possibly Unbounded Hessian Approximations}%
\label{sec:B-unbounded}


\section{Special Case: Diagonal Hessian Approximations}%
\label{sec:dh}

We now consider the case where \(B_k\) may not be bounded and formulate a variant of \citep[Lemma~\(1\)]{leconte-orban-2023}.

\section{Implementation and Numerical Results}%
\label{sec:numerical}

We compare several variants:
\begin{itemize}
  \item R2
  \item TR-R2
  \item TRDH (maybe? Include the updates of \citet{zhu-nazareth-wolkowicz-1999})
  \item TR-TRDH (maybe? Include the updates of \citet{zhu-nazareth-wolkowicz-1999})
  \item RN2-R2
  \item RN2DH (several variants: spectral, safeguarded PSB, safeguarded Andrei, those of \citet{zhu-nazareth-wolkowicz-1999})
  \item RN2-RN2DH (same variants).
\end{itemize}

\section{Discussion}%
\label{sec:discussion}


%\begin{acknowledgements}
%If you'd like to thank anyone, place your comments here
%and remove the percent signs.
%\end{acknowledgements}

%% References
%% We use Natbib, which is accepted by SIAM and Springer journals
%% see the cheat sheet: http://merkel.texture.rocks/Latex/natbib.php

% Do not use \cite{}. Only use the Natbib commands.
% \citet{jon90}                   -->     Jones et al. (1990)
% \citet[chap. 2]{jon90}          -->     Jones et al. (1990, chap. 2)
% \citep{jon90}                   -->     (Jones et al., 1990)
% \citep[chap. 2]{jon90}          -->     (Jones et al., 1990, chap. 2)
% \citep[see][]{jon90}            -->     (see Jones et al., 1990)
% \citep[see][chap. 2]{jon90}     -->     (see Jones et al., 1990, chap. 2)
% \citet*{jon90}                  -->     Jones, Baker, and Williams (1990)
% \citep*{jon90}                  -->     (Jones, Baker, and Williams, 1990)

%% References with BibTeX database:
% - each reference should have a DOI
% - use the strings provided for the journal name

\small
\bibliographystyle{abbrvnat}
\bibliography{abbrv,rn2}
\normalsize

\appendix
\clearpage

\section{Instructions}%
\label{sec:instructions}

Your \LaTeX\ source will be managed by Git.
Please write \emph{exactly one sentence per line}, as that will make it much easier to visualize diffs when you make changes.

Your co-authors will have to read it.
Therefore, it should be treated exactly as you treat code.
That means:
\begin{enumerate}
  \item use comments where appropriate;
  \item indent text and \LaTeX\ commands inside environments;
  \item use spaces between commands.
\end{enumerate}

\section{Another Section}%
\label{sec:more-stuff}

We use Natbib, which means that we do not use the \verb|\cite{}| command.
Instead, we only use Natbib commands, such as \verb|\citep{}| and \verb|\citet{}|.
Please refer to the Natbib cheat sheet at \http{merkel.texture.rocks/Latex/natbib.php}.

\verb|\citep{}| produces \citep{aravkin-baraldi-orban-2022} while \verb|\citet{}| produces \citet{aravkin-baraldi-orban-2022}.

\subsection{Subsection title}%
\label{sec:2}

Don't forget to give each section and subsection a unique label (see \cref{sec:more-stuff}).

\begin{lemma}%
  \label{lem:important}
  This is a lemma.
\end{lemma}

\begin{proof}
  The proof is immediate.
  \qed
\end{proof}

Use \verb|\Cref{}| to refer to environments, including lemmas, theorems, definitions, and sections, which causes the name of the environment and its number to be hyperlinked together.

\begin{theorem}%
  \label{thm:important}
  This theorem relies on \Cref{lem:important}.
\end{theorem}

\begin{proof}
  If a proof ends with a displayed equation, it's possible to place the ``end of proof'' symbol at the end of the equation with \verb|\tag*{\qed}|:
  \[
    e = mc^2.
    \tag*{\qed}
  \]
\end{proof}

\begin{corollary}
  This corollary follows directly from \Cref{thm:important}.
\end{corollary}

\paragraph{Paragraph headings}

Use paragraph headings as needed.
\begin{equation}%
  \label{eq:pythagoras}
  a^2+b^2=c^2.
\end{equation}

However, do not use \verb|\Cref{}| to refer to equations as that would result in \Cref{eq:pythagoras}.
Instead, use \verb|\eqref{}| preceded by an unbreakable space: refer to~\eqref{eq:pythagoras}.

% For one-column wide figures use
% \begin{figure}
% % Use the relevant command to insert your figure file.
% % For example, with the graphicx package use
%   \includegraphics{example.eps}
% % figure caption is below the figure
% \caption{Please write your figure caption here}
% \label{fig:1}       % Give a unique label
% \end{figure}
% %
% % For two-column wide figures use
% \begin{figure*}
% % Use the relevant command to insert your figure file.
% % For example, with the graphicx package use
%   \includegraphics[width=0.75\textwidth]{example.eps}
% % figure caption is below the figure
% \caption{Please write your figure caption here}
% \label{fig:2}       % Give a unique label
% \end{figure*}
%
% For tables use
\begin{table}[ht]
  % table caption is above the table
  \caption{Please write your table caption here}
  \label{tab:1}       % Give a unique label
  % For LaTeX tables use
  \begin{tabular}{lll}
    \hline\noalign{\smallskip}
    first & second & third  \\
    \noalign{\smallskip}\hline\noalign{\smallskip}
    number & number & number \\
    number & number & number \\
    \noalign{\smallskip}\hline
  \end{tabular}
\end{table}

\newpage

\hypertarget{contents}{}  % so clicking on [toc] in the header leads here
\tableofcontents
\listoftodos

\end{document}
% end of file template.tex

